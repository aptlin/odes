\documentclass[12pt,letterpaper]{article}
\usepackage{fullpage}
\usepackage[top=2cm, bottom=4.5cm, left=2.5cm, right=2.5cm]{geometry}
\usepackage{amsmath,amsthm,amsfonts,amssymb,amscd}
\usepackage{lastpage}
\usepackage{enumerate}
\usepackage{enumitem}
\usepackage{fancyhdr}
\usepackage{mathrsfs}
\usepackage{xcolor}
\usepackage{graphicx}
\usepackage{listings}
\usepackage{hyperref}

\hypersetup{%
  colorlinks=true,
  linkcolor=blue,
  linkbordercolor={0 0 1}
}

\renewcommand\lstlistingname{Algorithm}
\renewcommand\lstlistlistingname{Algorithms}
\def\lstlistingautorefname{Alg.}
\DeclareMathOperator{\Tr}{tr}
\lstdefinestyle{Python}{
    language        = Python,
    frame           = lines,
    basicstyle      = \footnotesize,
    keywordstyle    = \color{blue},
    stringstyle     = \color{green},
    commentstyle    = \color{red}\ttfamily
}

\setlength{\parindent}{0.0in}
\setlength{\parskip}{0.05in}

% Edit these as appropriate
\newcommand\hwnumber{1}                  % <-- homework number
\newcommand\NetIDa{Sasha Illarionov}           % <-- NetID of person #1
% \newcommand\NetIDb{netid12038}           % <-- NetID of person #2 (Comment this line out for problem sets)

\pagestyle{fancyplain}
\headheight 35pt
\lhead{\NetIDa}
\rhead{\today}
\lfoot{}
\cfoot{}
\rfoot{\small\thepage}
\headsep 1.5em

\begin{document}

\section*{Problem 1}

\begin{enumerate}[label=(\alph*)]
	\item
	      Consider the following system of equations:
	      \begin{align}
		      \begin{pmatrix}
			      \dot{x} \\
			      \dot{y}
		      \end{pmatrix} & =
		      \begin{pmatrix}
			      1 & -5 \\
			      2 & -5
		      \end{pmatrix} \begin{pmatrix}
			      x \\
			      y
		      \end{pmatrix} \\ \label{eq:init_1}
		                                & =
		      M \begin{pmatrix}
			      x \\
			      y
		      \end{pmatrix}\end{align}

	      Let's compute the eigenvalues of the matrix:

	      \begin{align}
		      \det \begin{pmatrix}
			      1 - \lambda & -5           \\
			      2           & -5 - \lambda
		      \end{pmatrix} & = (\lambda + 5)(\lambda - 1) + 10               \\
		                                     & = \lambda ^ 2 + 4 \lambda + 5 \label{eq:char_1}
	      \end{align}

	      Therefore, $\lambda = -2 + i$ is an eigenvalue.\vspace{1in}\\
	      Moreover, $w = \begin{pmatrix} 3 + i \\ 2 \end{pmatrix}$
	      is a corresponding eigenvector, since
	      \begin{align*}
		      (M - \lambda I)w & =
		      \begin{pmatrix}
			      3 - i & -5     \\
			      2     & -3 - i
		      \end{pmatrix} \begin{pmatrix} 3 + i \\ 2 \end{pmatrix}\\
		                       & = \begin{pmatrix}
			      10 - 10         \\
			      2(3+i) - 2(3+i)
		      \end{pmatrix}  \\
		                       & = \begin{pmatrix}
			      0 \\0
		      \end{pmatrix}
	      \end{align*}

	\item Note that $\overline{\lambda} = -2 - i$ is also a solution of the Equation \ref{eq:char_1}, implying that $\overline{\lambda}$ is also an eigenvalue. Moreover, since $\overline{(M-\lambda I)w} = (\overline{M} - \overline(\lambda I))\overline{w} = (M - \overline{\lambda} I) \overline{w}$, then $\overline{w} =  \begin{pmatrix}
			      3-i \\
			      2
		      \end{pmatrix}$ is another eigenvector of $M$.

	\item Now, consider the change of coordinates matrix $C = \begin{pmatrix}
			      3 & -1 \\
			      2 & 0
		      \end{pmatrix}$, so that $\begin{pmatrix}
			      x \\
			      y
		      \end{pmatrix} = C \begin{pmatrix}
			      u \\
			      v
		      \end{pmatrix}$.

	      Note that $\begin{pmatrix}
			      \dot{x} \\
			      \dot{y}
		      \end{pmatrix} = C \begin{pmatrix}
			      \dot{u} \\
			      \dot{v}
		      \end{pmatrix}$, and thus
	      \begin{align}
		      C\begin{pmatrix}
			      \dot{u} \\
			      \dot{v}
		      \end{pmatrix} & =
		      M C \begin{pmatrix}
			      \dot{u} \\
			      \dot{v}
		      \end{pmatrix},
	      \end{align} which implies that
	      \begin{align}
		      \begin{pmatrix}
			      \dot{u} \\
			      \dot{v}
		      \end{pmatrix} & =
		      C^{-1}M C \begin{pmatrix}
			      u \\
			      v
		      \end{pmatrix}\\
		                                 & =
		      \frac{1}{2}\begin{pmatrix}
			      0  & 1 \\
			      -2 & 3
		      \end{pmatrix} \begin{pmatrix}
			      1 & -5 \\
			      2 & -5
		      \end{pmatrix}
		      \begin{pmatrix}
			      3 & -1 \\
			      2 & 0
		      \end{pmatrix}\begin{pmatrix}
			      u \\
			      v
		      \end{pmatrix}\\
		                                 & =
		      \frac{1}{2}\begin{pmatrix}
			      2 & -5 \\
			      4 & -5
		      \end{pmatrix}
		      \begin{pmatrix}
			      3 & -1 \\
			      2 & 0
		      \end{pmatrix}\begin{pmatrix}
			      u \\
			      v
		      \end{pmatrix}\\
		                                 & =
		      \frac{1}{2}\begin{pmatrix}
			      -4 & -2 \\
			      2  & -4
		      \end{pmatrix}\begin{pmatrix}
			      u \\
			      v
		      \end{pmatrix}\\
		                                 & =
		      \begin{pmatrix}
			      -2 & -1 \\
			      1  & -2
		      \end{pmatrix}\begin{pmatrix}
			      u \\
			      v
		      \end{pmatrix} \label{eq:orthog_1}
	      \end{align}
	\item Consider the following equation:
	      \begin{equation}
		      \dot{z} = \lambda z \label{eq:complex_1}.
	      \end{equation}

	      Set $z = u + i v$ for real differentiable functions $u$ and $v$.
	      Similarly, suppose that $\lambda = \alpha + i \beta$ for real scalars $\alpha$ and $\beta$.

	      Note that
	      \begin{align}
		      \dot{u} + i \dot{v} & = (\alpha + i \beta) (u + i v)                 \\
		                          & = \alpha u - \beta v + i (\beta u + \alpha v),
	      \end{align} so the equation \ref{eq:complex_1} becomes
	      \begin{align}
		      \begin{pmatrix}
			      \dot{u} \\
			      \dot{v}
		      \end{pmatrix} & = \begin{pmatrix}
			      \alpha & -\beta \\
			      \beta  & \alpha
		      \end{pmatrix} \begin{pmatrix}
			      u \\
			      v
		      \end{pmatrix} \\
		                                 & =
		      \begin{pmatrix}
			      -2 & -1 \\
			      1  & -2
		      \end{pmatrix} \begin{pmatrix}
			      u \\
			      v
		      \end{pmatrix} \label{eq:complexed}
	      \end{align}

	\item Equation \ref{eq:complexed} coincides with Equation \ref{eq:orthog_1}.
	\item Notice that all solutions of Equation $\ref{eq:complex_1}$ are in the form $z = z(0) \exp(\lambda t)$.
	\item Let's find the real and imaginary components of the solution above. Note the following:
	      \begin{align}
		      z = u +  i v & = (u(0)  + i v(0)) \exp((\alpha + i \beta)t)                                                        \\
		                   & =\exp(\alpha t)(u(0)  + i v(0))(\cos(\beta t) + i \sin(\beta t))                                    \\
		                   & = \exp(\alpha t)(u(0)\cos(\beta t) - v(0)\sin(\beta t) + i (u(0)\sin(\beta t) + v(0)\cos(\beta t)))
	      \end{align}

	      Hence, $u(t) = \exp(\alpha t)(u(0)\cos(\beta t) - v(0)\sin(\beta t)) = \exp(\alpha t)\begin{pmatrix}
			      \cos(\beta t) & -\sin(\beta t)
		      \end{pmatrix} \begin{pmatrix}
			      u(0) \\
			      v(0)
		      \end{pmatrix}$, and $v(t) = \exp(\alpha t) \begin{pmatrix}
			      \sin(\beta t) & \cos(\beta t)
		      \end{pmatrix} \begin{pmatrix}
			      u(0) \\
			      v(0)
		      \end{pmatrix}$
	\item Since we have shown already that real solutions of Equation \ref{eq:orthog_1} coincide with the real and imaginary components of complex solutions to Equation \ref{eq:complexed}, then, given that $\alpha = -2$ and $\beta = 1$, we obtain
	      \begin{align}
		      \begin{pmatrix}
			      u(t) \\
			      v(t)
		      \end{pmatrix} & =
		      \exp(\alpha t)\begin{pmatrix}
			      \cos(\beta t) & -\sin(\beta t) \\
			      \sin(\beta t) & \cos(\beta t)
		      \end{pmatrix}  \begin{pmatrix}
			      u(0) \\
			      v(0)
		      \end{pmatrix} \\
		                                 & =
		      \exp(-2 t)\begin{pmatrix}
			      \cos(t) & -\sin(t) \\
			      \sin(t) & \cos(t)
		      \end{pmatrix}  \begin{pmatrix}
			      u(0) \\
			      v(0)
		      \end{pmatrix}
	      \end{align}
	\item To find the solutions to Equation \ref{eq:init_1}, we return to the original coordinates:

	      \begin{align}
		      \begin{pmatrix}
			      x(t) \\
			      y(t)
		      \end{pmatrix} & = C \begin{pmatrix}
			      u(t) \\
			      v(t)
		      \end{pmatrix}                                                    \\
		                                 & = \begin{pmatrix}
			      3 & -1 \\
			      2 & 0
		      \end{pmatrix}
		      \exp(-2 t)\begin{pmatrix}
			      \cos(t) & -\sin(t) \\
			      \sin(t) & \cos(t)
		      \end{pmatrix}  \begin{pmatrix}
			      u(0) \\
			      v(0)
		      \end{pmatrix}\\
		                                 & = \exp(-2 t) \begin{pmatrix}
			      3\cos(t) - \sin(t) & -3\sin(t) - \cos(t) \\
			      2\cos(t)           & -2\sin(t)
		      \end{pmatrix}\begin{pmatrix}
			      u(0) \\
			      v(0)
		      \end{pmatrix}\label{eq:sol_1}
	      \end{align}
	\item Set $w = a + i b$, where $a = \begin{pmatrix}
			      3 \\
			      2
		      \end{pmatrix}$ and $b = \begin{pmatrix}
			      1 \\
			      0
		      \end{pmatrix}$. Note the following:
	      \begin{align}
		      w\exp(\lambda t) & = w \exp((-2 + i)t)                                        \\
		                       & =\exp(-2t) (a + ib)(\cos(t) + i\sin(t))                    \\
		                       & = \exp(-2t)(a\cos(t) - b\sin(t) + i(a\sin(t) + b\cos(t))),
	      \end{align}
	      so
	      \begin{align}
		      \Re(w \exp(\lambda t)) & = \exp(-2t)(a\cos(t) - b\sin(t)) \\
		      \Im(w \exp(\lambda t)) & = \exp(-2t)(a\sin(t) + b\cos(t))
	      \end{align}
	      Then Equation $\ref{eq:sol_1}$ can be written as
	      \begin{align}
		      \begin{pmatrix}
			      x(t) \\
			      y(t)
		      \end{pmatrix} & = \exp(-2 t) \begin{pmatrix}
			      3\cos(t) - \sin(t) & -3\sin(t) - \cos(t) \\
			      2\cos(t)           & -2\sin(t)
		      \end{pmatrix}\begin{pmatrix}
			      u(0) \\
			      v(0)
		      \end{pmatrix}                                   \\
		                                 & = u(0) \exp(-2t)\left(\cos(t)\begin{pmatrix}
				      3 \\
				      2
			      \end{pmatrix} - \sin(t) \begin{pmatrix}
				      1 \\
				      0
			      \end{pmatrix}\right) \\
		                                 & - v(0) \exp(-2t)\left(\sin(t)\begin{pmatrix}
				      3 \\
				      2
			      \end{pmatrix}+ \cos(t)\begin{pmatrix}
				      1 \\
				      0
			      \end{pmatrix}\right)   \\
		                                 & = u(0) \Re(w \exp(\lambda t)) - v(0)\Im(w \exp(\lambda t))
	      \end{align}
	\item See the notebook.
\end{enumerate}

\section*{Problem 2}

See the notebook for phase portraits.

\begin{enumerate}[label=(\alph*)]
	\item Consider the following system of equations $M$:
	      \begin{align}
		      \begin{pmatrix}
			      \dot{x} \\
			      \dot{y}
		      \end{pmatrix} & = \begin{pmatrix}
			      -2 & 1  \\
			      0  & -2
		      \end{pmatrix}
		      \begin{pmatrix}
			      x \\
			      y
		      \end{pmatrix}
	      \end{align}
	      Notice that the matrix is a Jordan block corresponding to the eigenvalue $\lambda = -2$.
	      Then the solution is in the form of a matrix exponent, which we compute as follows:
	      \begin{align}
		      z(t) & = \exp(Mt)z_0                                                            \\
		           & = \exp\left(\left(-2 I + \begin{pmatrix}
					      0 & 1 \\
					      0 & 0
				      \end{pmatrix}\right)t\right)z_0    \\
		           & = \exp(-2t I) \left(I + \frac{1}{1!}\begin{pmatrix}
			      0 & t \\
			      0 & 0
		      \end{pmatrix}\right)z_0 \\
		           & = \exp(-2t)\begin{pmatrix}
			      1 & t \\
			      0 & 1
		      \end{pmatrix}z_0
	      \end{align}
	      Hence, the solutions are in the form
	      \begin{align}
		      \begin{pmatrix}
			      x(t) \\
			      y(t)
		      \end{pmatrix} & =\exp(-2t)\begin{pmatrix}
			      1 & t \\
			      0 & 1
		      \end{pmatrix}\begin{pmatrix}
			      x(0) \\
			      y(0)
		      \end{pmatrix}
	      \end{align}
	      The stationary point is a stable dicritic node.

	\item Consider the following system of equations $M$:
	      \begin{align}
		      \begin{pmatrix}
			      \dot{x} \\
			      \dot{y}
		      \end{pmatrix} & = \begin{pmatrix}
			      12 & 12 \\
			      -9 & -9
		      \end{pmatrix}
		      \begin{pmatrix}
			      x \\
			      y
		      \end{pmatrix}
	      \end{align}
	      Note that it is degenerate, so there exist no non-degenerate critical points.
	      Let's compute the characteristic polynomial of $M$:
	      \begin{align}
		      (12 - \lambda)(-9 -  \lambda) + 9 \cdot 12 & =0  \\
		      \Leftrightarrow \lambda ^2 -3\lambda       & = 0
	      \end{align}
	      Therefore, $\lambda = 0$ and $\lambda =3$ are eigenvalues, with $\begin{pmatrix}
			      1  \\
			      -1
		      \end{pmatrix}$ and $\begin{pmatrix}
			      4  \\
			      -3
		      \end{pmatrix}$ as corresponding eigenvectors.

	      Changing the coordinates, we obtain the following system of equations:

	      \begin{align}
		      \begin{pmatrix}
			      \dot{u} \\
			      \dot{v}
		      \end{pmatrix} & = \begin{pmatrix}
			      0 & 0 \\
			      0 & 3
		      \end{pmatrix}\begin{pmatrix}
			      u \\
			      v
		      \end{pmatrix},
	      \end{align}
	      which has solutions in the form:

	      \begin{align}
		      \begin{pmatrix}
			      u \\
			      v
		      \end{pmatrix} & =
		      \begin{pmatrix}
			      0 & 0        \\
			      0 & \exp(3t)
		      \end{pmatrix} \begin{pmatrix}
			      u(0) \\
			      v(0)
		      \end{pmatrix}
	      \end{align}

	      Changing the coordinates back, we obtain:

	      \begin{align}
		      \begin{pmatrix}
			      x \\
			      y
		      \end{pmatrix} & =
		      \begin{pmatrix}
			      1  & 4  \\
			      -1 & -3
		      \end{pmatrix}\begin{pmatrix}
			      0 & 0        \\
			      0 & \exp(3t)
		      \end{pmatrix}
		      \begin{pmatrix}
			      -3 & -4 \\
			      1  & 1
		      \end{pmatrix}
		      \begin{pmatrix}
			      x(0) \\
			      y(0) \\
		      \end{pmatrix}\\
		                                 & =
		      \begin{pmatrix}
			      0 & 4 \exp(3t) \\
			      0 & -3\exp(3t)
		      \end{pmatrix}
		      \begin{pmatrix}
			      -3 & -4 \\
			      1  & 1
		      \end{pmatrix}
		      \begin{pmatrix}
			      x(0) \\
			      y(0) \\
		      \end{pmatrix}\\
		                                 & =
		      \begin{pmatrix}
			      4\exp(3t)  & 4\exp(3t)  \\
			      -3\exp(3t) & -3\exp(3t)
		      \end{pmatrix}
		      \begin{pmatrix}
			      x(0) \\
			      y(0) \\
		      \end{pmatrix}\\
		                                 & =
		      \exp(3t)\begin{pmatrix}
			      4  & 4  \\
			      -3 & -3
		      \end{pmatrix}
		      \begin{pmatrix}
			      x(0) \\
			      y(0) \\
		      \end{pmatrix}
	      \end{align}

	\item Consider the following system of equations $M$:
	      \begin{align}
		      \begin{pmatrix}
			      \dot{x} \\
			      \dot{y}
		      \end{pmatrix} & = \begin{pmatrix}
			      -2 & -13 \\
			      2  & 8
		      \end{pmatrix}
		      \begin{pmatrix}
			      x \\
			      y
		      \end{pmatrix}
	      \end{align}.

	      Let's compute the characteristic polynomial:

	      \begin{align}
		      (-2 - \lambda)(8-\lambda) + 26  & = 0 \\
		      (\lambda + 2)(\lambda - 8) + 26 & = 0 \\
		      \lambda^2 -6 \lambda + 10       & = 0
	      \end{align}
	      Then the eigenvalues are $3 + i$ and  $3 - i$.

	      Set $\lambda = 3 + i$.
	      Now we find the eigenvector $v = \begin{pmatrix}
			      a \\
			      b
		      \end{pmatrix}$:
	      \begin{align}
		                      & \begin{pmatrix}
			      -5 - i & -13   \\
			      2      & 5 - i
		      \end{pmatrix}v   = 0                          \\
		      \Leftrightarrow & \begin{pmatrix}
			      -5a - 13b -ai \\
			      2a + 5b - bi
		      \end{pmatrix} v  =\begin{pmatrix}
			      0 \\
			      0
		      \end{pmatrix} \\
		      \Leftrightarrow & \begin{pmatrix}
			      -2(5 + i)a - 26b \\
			      2(5 + i)a + 26b
		      \end{pmatrix} v  =\begin{pmatrix}
			      0 \\
			      0
		      \end{pmatrix}
	      \end{align}
	      Therefore, $v= \begin{pmatrix}
			      13     \\
			      -5 - i
		      \end{pmatrix}$ is an eigenvector.
	      By the derivation in Problem 1, the real solutions are in the following form (for $\alpha = 3$ and $\beta = 1$):

	      \begin{align}
		      \begin{pmatrix}
			      x \\
			      y
		      \end{pmatrix} & =
		      \begin{pmatrix}
			      13 & 0 \\
			      -5 & 1
		      \end{pmatrix}
		      \exp(3 t)\begin{pmatrix}
			      \cos( t) & -\sin( t) \\
			      \sin( t) & \cos( t)
		      \end{pmatrix}
		      \frac{1}{13}
		      \begin{pmatrix}
			      1 & 0  \\
			      5 & 13
		      \end{pmatrix}
		      \begin{pmatrix}
			      x(0) \\
			      y(0)
		      \end{pmatrix}\\
		                                  & =
		      \frac{\exp(3t)}{13}
		      \begin{pmatrix}
			      13 \cos(t)           & 13\sin(t)          \\
			      -5 \cos(t) + \sin(t) & 5\sin(t) + \cos(t)
		      \end{pmatrix}
		      \begin{pmatrix}
			      1 & 0  \\
			      5 & 13
		      \end{pmatrix}
		      \begin{pmatrix}
			      x(0) \\
			      y(0)
		      \end{pmatrix}\\
		                                  & =
		      \frac{\exp(3t)}{13}
		      \begin{pmatrix}
			      13\cos(t) + 65\sin(t) & 169 \sin(t)           \\
			      26\sin(t)             & 65\sin(t) + 13\cos(t)
		      \end{pmatrix}
		      \begin{pmatrix}
			      x(0) \\
			      y(0)
		      \end{pmatrix}\\
		                                  & =
		      \exp(3t)
		      \begin{pmatrix}
			      \cos(t) + 5\sin(t) & 13 \sin(t)         \\
			      2\sin(t)           & 5\sin(t) + \cos(t)
		      \end{pmatrix}
		      \begin{pmatrix}
			      x(0) \\
			      y(0)
		      \end{pmatrix}
	      \end{align}
	      The critical point is an unstable focus.
	\item Consider the following system of equations $M$:
	      \begin{align}
		      \begin{pmatrix}
			      \dot{x} \\
			      \dot{y}
		      \end{pmatrix} & = \begin{pmatrix}
			      10  & 7   \\
			      -14 & -11
		      \end{pmatrix}
		      \begin{pmatrix}
			      x \\
			      y
		      \end{pmatrix}
	      \end{align}
	      Let's compute the characteristic polynomial:
	      \begin{align}
		      \lambda^2 + \lambda + (-110 + 98) & = 0 \\
		      \lambda^2 + \lambda + -12         & = 0
	      \end{align}
	      Then $-4$ and $3$ are eigenvalues.
	      For $\lambda = -4$, the eigenvector $u$ is such that
	      \begin{align}
		      \begin{pmatrix}
			      14  & 7  \\
			      -14 & -7
		      \end{pmatrix}u = 0.
	      \end{align}
	      Set $u = \begin{pmatrix}
			      1  \\
			      -2
		      \end{pmatrix}$.

	      For $\lambda = 3$, the eigenvector $v$ is such that
	      \begin{align}
		      \begin{pmatrix}
			      7   & 7   \\
			      -14 & -14
		      \end{pmatrix}u = 0.
	      \end{align}
	      Set $v = \begin{pmatrix}
			      1  \\
			      -1
		      \end{pmatrix}$.

	      Changing the basis to that of eigenvectors, we obtain the following system of equations:
	      \begin{align}
		      \begin{pmatrix}
			      \dot{q} \\
			      \dot{r}
		      \end{pmatrix} & =
		      \begin{pmatrix}
			      -4 & 0 \\
			      0  & 3
		      \end{pmatrix}
		      \begin{pmatrix}
			      q \\
			      r
		      \end{pmatrix},
	      \end{align}
	      which has solutions in the form:
	      \begin{align}
		      \begin{pmatrix}
			      q \\
			      r
		      \end{pmatrix} & =
		      \begin{pmatrix}
			      \exp(-4t) & 0        \\
			      0         & \exp(3t)
		      \end{pmatrix}
		      \begin{pmatrix}
			      q_0 \\
			      r_0
		      \end{pmatrix}.
	      \end{align}

	      Changing the coordinates back, we obtain:
	      \begin{align}
		      \begin{pmatrix}
			      x \\
			      y
		      \end{pmatrix} & =
		      \begin{pmatrix}
			      1  & 1  \\
			      -2 & -1
		      \end{pmatrix}
		      \begin{pmatrix}
			      \exp(-4t) & 0        \\
			      0         & \exp(3t)
		      \end{pmatrix}
		      \begin{pmatrix}
			      -1 & -1 \\
			      2  & 1
		      \end{pmatrix}
		      \begin{pmatrix}
			      x_0 \\
			      y_0
		      \end{pmatrix} \\
		                                  & =
		      \begin{pmatrix}
			      \exp(-4t)   & \exp(3t)  \\
			      -2\exp(-4t) & -\exp(3t)
		      \end{pmatrix}
		      \begin{pmatrix}
			      -1 & -1 \\
			      2  & 1
		      \end{pmatrix}
		      \begin{pmatrix}
			      x_0 \\
			      y_0
		      \end{pmatrix}\\
		                                  & =
		      \begin{pmatrix}
			      -\exp(-4t) + 2\exp(3t)  & -\exp(-4t) + \exp(3t) \\
			      2 \exp(-4t) -2 \exp(3t) & 2\exp(-4t) - \exp(3t)
		      \end{pmatrix}
		      \begin{pmatrix}
			      x_0 \\
			      y_0
		      \end{pmatrix}
	      \end{align}
	      The critical point is a saddle.
	\item Consider the following system of equations $M$:
	      \begin{align}
		      \begin{pmatrix}
			      \dot{x} \\
			      \dot{y}
		      \end{pmatrix} & = \begin{pmatrix}
			      5 & -4 \\
			      3 & -2
		      \end{pmatrix}
		      \begin{pmatrix}
			      x \\
			      y
		      \end{pmatrix}
	      \end{align}

	      Consider the characteristic polynomial of $M$:

	      \begin{align}
		      \lambda^2 - 3 \lambda + 2 & = 0
	      \end{align}

	      Then $1$ and $2$ are eigenvalues.

	      For $\lambda = 1$, the corresponding eigenvector $u$ satisfies the following:

	      \begin{align}
		      \begin{pmatrix}
			      4 & -4 \\
			      3 & -3
		      \end{pmatrix}u & =
		      0.
	      \end{align}

	      Set $u=\begin{pmatrix}
			      1 \\
			      1
		      \end{pmatrix}$.


	      For $\lambda = 2$, the corresponding eigenvector $v$ satisfies the following:

	      \begin{align}
		      \begin{pmatrix}
			      3 & -4 \\
			      3 & -4
		      \end{pmatrix}v & =
		      0.
	      \end{align}

	      Set $v=\begin{pmatrix}
			      4 \\
			      3
		      \end{pmatrix}$.

	      Changing the basis to that of eigenvectors, we obtain the following system of equations:
	      \begin{align}
		      \begin{pmatrix}
			      \dot{q} \\
			      \dot{r}
		      \end{pmatrix} & =
		      \begin{pmatrix}
			      1 & 0 \\
			      0 & 2
		      \end{pmatrix}
		      \begin{pmatrix}
			      q \\
			      r
		      \end{pmatrix},
	      \end{align}
	      which has solutions in the form:
	      \begin{align}
		      \begin{pmatrix}
			      q \\
			      r
		      \end{pmatrix} & =
		      \begin{pmatrix}
			      \exp(t) & 0        \\
			      0       & \exp(2t)
		      \end{pmatrix}
		      \begin{pmatrix}
			      q_0 \\
			      r_0
		      \end{pmatrix}.
	      \end{align}

	      Changing the coordinates back, we obtain:
	      \begin{align}
		      \begin{pmatrix}
			      x \\
			      y
		      \end{pmatrix} & =
		      \begin{pmatrix}
			      1 & 4 \\
			      1 & 3
		      \end{pmatrix}
		      \begin{pmatrix}
			      \exp(t) & 0        \\
			      0       & \exp(2t)
		      \end{pmatrix}
		      (-1)\begin{pmatrix}
			      3  & -4 \\
			      -1 & 1
		      \end{pmatrix}
		      \begin{pmatrix}
			      x_0 \\
			      y_0
		      \end{pmatrix} \\
		                                  & =
		      -\begin{pmatrix}
			      \exp(t) & 4\exp(2t) \\
			      \exp(t) & 3\exp(2t)
		      \end{pmatrix}
		      \begin{pmatrix}
			      3  & -4 \\
			      -1 & 1
		      \end{pmatrix}
		      \begin{pmatrix}
			      x_0 \\
			      y_0
		      \end{pmatrix}\\
		                                  & =
		      -\begin{pmatrix}
			      3\exp(t) - 4\exp(2t)  & -4\exp(t) + 4\exp(2t) \\
			      3 \exp(t) -3 \exp(2t) & -4\exp(t) + 3\exp(2t)
		      \end{pmatrix}
		      \begin{pmatrix}
			      x_0 \\
			      y_0
		      \end{pmatrix}
	      \end{align}
	      The critical point is an unstable node.
\end{enumerate}
\section*{Problem 3}
\begin{enumerate}[label=(\alph*)]
	\item
	      Consider the following system of equations parametrised by $s \in \mathbb{R}$:

	      \begin{align}
		      \begin{pmatrix}
			      \dot{x} \\
			      \dot{y}
		      \end{pmatrix} & =
		      \begin{pmatrix}
			      -6 & -7 \\
			      s  & 4
		      \end{pmatrix}
		      \begin{pmatrix}
			      x \\
			      y
		      \end{pmatrix}.
	      \end{align}

	      Notice that the corresponding characteristic polynomial is
	      \begin{align}
		      \lambda^2 +  2\lambda + (-24 + 7s) & = 0.
	      \end{align}

	      Now, the discriminant of the quadratic is $\Delta = 4  -4 (7s -24) = 100 - 28s$.

	      If $s = \frac{25}{7}$, then there is only one eigenvalue ($\lambda = -1$), and since the corresponding matrix is not diagonal, then the critical point is a stable degenerate node.

	      If $s > \frac{25}{7}$, then $\Delta < 0$, and thus the critical value is a stable focus (since $\Tr M < 0, \det M > \frac{(\Tr M)^2} {2}$), and the matrix has two complex eigenvalues having a negative real component.

	      If $s< \frac{25}{7}$, but $\sqrt{\Delta} < 2$, which is equivalent to $s > \frac{24}{7}$, then $\det M > 0$, and the critical  point is a stable node, with the matrix having two  real negative eigenvalues.

	      If, on the other hand, $\sqrt{\Delta} > 2$, so that $s < \frac{24}{7}$, then the critical point is a saddle, with the matrix having two real eigenvalues of opposite sign.

	      Otherwise, if $s = \frac{24}{7}$, then the matrix is degenerate. There are no non-degenerate critical points, and the matrix has two real eigenvalues, one zero and one negative.


	\item

	      If $s < \frac{24}{7}$, then the critical point $(0,0)$ is neither asymptotically nor Lyapunov stable.

	      If $s = \frac{24}{7} \approx 3.4286$, then the critical point is unstable as well, since in this case the phase space consists of straight lines, which do not 'concentrate' around $(0,0)$.

	      If $s > \frac{24}{7}$, then the critical point is both Lyapunov and asymptotically stable, since the real components of eigenvalues are negative.

	\item See the notebook for phase portraits.
\end{enumerate}

\section*{Problem 4}
Consider the following system of equations:

\begin{align}
	\begin{pmatrix}
		\dot{x} \\
		\dot{y}
	\end{pmatrix} & =
	\begin{pmatrix}
		\sin(x) + \exp(y) - 1 \\
		\sin(x - y)
	\end{pmatrix}
\end{align}
\begin{enumerate}[label=(\alph*)]
	\item See the notebook.
	\item
	      Let's compute the Jacobian matrix at (0,0):

	      \begin{align}
		      J(0,0) & =
		      \begin{pmatrix}
			      \frac{\partial \dot{x}}{\partial x} & \frac{\partial \dot{x}}{\partial y} \\
			      \frac{\partial \dot{y}}{\partial x} & \frac{\partial \dot{y}}{\partial y}
		      \end{pmatrix} \mid_{(0,0)}\\
		             & = \begin{pmatrix}
			      \cos(x)   & \exp{y}    \\
			      \cos(x-y) & -\cos(x-y)
		      \end{pmatrix} \mid_{(0,0)} \\
		             & = \begin{pmatrix}
			      1 & 1  \\
			      1 & -1
		      \end{pmatrix}
	      \end{align}

	      The corresponding linear system of equations is thus:

	      \begin{align}
		      \begin{pmatrix}
			      \dot{u} \\
			      \dot{v}
		      \end{pmatrix}
		        & =
		      \begin{pmatrix}
			      1 & 1  \\
			      1 & -1
		      \end{pmatrix}
		      \begin{pmatrix}
			      u \\
			      v
		      \end{pmatrix}
	      \end{align}
	\item See the notebook.
	\item See the notebook.
\end{enumerate}

\end{document}
